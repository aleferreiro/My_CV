%!TEX TS-program = xelatex
%!TEX encoding = UTF-8 Unicode
% Awesome CV LaTeX Template for CV/Resume
%
% This template has been downloaded from:
% https://github.com/posquit0/Awesome-CV
%
% Author:
% Claud D. Park <posquit0.bj@gmail.com>
% http://www.posquit0.com
%
%
% Adapted to be an Rmarkdown template by Mitchell O'Hara-Wild
% 23 November 2018
%
% Template license:
% CC BY-SA 4.0 (https://creativecommons.org/licenses/by-sa/4.0/)
%
%-------------------------------------------------------------------------------
% CONFIGURATIONS
%-------------------------------------------------------------------------------
% A4 paper size by default, use 'letterpaper' for US letter
\documentclass[11pt,a4paper,]{awesome-cv}

% Configure page margins with geometry
\usepackage{geometry}
\geometry{left=1.4cm, top=.8cm, right=1.4cm, bottom=1.8cm, footskip=.5cm}


% Specify the location of the included fonts
\fontdir[fonts/]

% Color for highlights
% Awesome Colors: awesome-emerald, awesome-skyblue, awesome-red, awesome-pink, awesome-orange
%                 awesome-nephritis, awesome-concrete, awesome-darknight

\definecolor{awesome}{HTML}{2b8cbe}

% Colors for text
% Uncomment if you would like to specify your own color
% \definecolor{darktext}{HTML}{414141}
% \definecolor{text}{HTML}{333333}
% \definecolor{graytext}{HTML}{5D5D5D}
% \definecolor{lighttext}{HTML}{999999}

% Set false if you don't want to highlight section with awesome color
\setbool{acvSectionColorHighlight}{true}

% If you would like to change the social information separator from a pipe (|) to something else
\renewcommand{\acvHeaderSocialSep}{\quad\textbar\quad}

\def\endfirstpage{\newpage}

%-------------------------------------------------------------------------------
%	PERSONAL INFORMATION
%	Comment any of the lines below if they are not required
%-------------------------------------------------------------------------------
% Available options: circle|rectangle,edge/noedge,left/right

\photo{img/profile.jpg}
\name{Alejandro Manuel Ferreiro}{}

\position{PhD fellow}
\address{Institute of Animal Diversity and Ecology (IDEA)}

\email{\href{mailto:aleferreiro7@gmail.com}{\nolinkurl{aleferreiro7@gmail.com}}}
\homepage{aleferreiro.netlify.app}
\orcid{0000-0002-0921-6137}
\github{aleferreiro}
\linkedin{alejandro-manuel-ferreiro-431aa4b0}
\twitter{aleferreiro1}

% \gitlab{gitlab-id}
% \stackoverflow{SO-id}{SO-name}
% \skype{skype-id}
% \reddit{reddit-id}


\usepackage{booktabs}

\providecommand{\tightlist}{%
	\setlength{\itemsep}{0pt}\setlength{\parskip}{0pt}}

%------------------------------------------------------------------------------



% Pandoc CSL macros
\newlength{\cslhangindent}
\setlength{\cslhangindent}{1.5em}
\newlength{\csllabelwidth}
\setlength{\csllabelwidth}{3em}
\newenvironment{CSLReferences}[3] % #1 hanging-ident, #2 entry spacing
 {% don't indent paragraphs
  \setlength{\parindent}{0pt}
  % turn on hanging indent if param 1 is 1
  \ifodd #1 \everypar{\setlength{\hangindent}{\cslhangindent}}\ignorespaces\fi
  % set entry spacing
  \ifnum #2 > 0
  \setlength{\parskip}{#2\baselineskip}
  \fi
 }%
 {}
\usepackage{calc}
\newcommand{\CSLBlock}[1]{#1\hfill\break}
\newcommand{\CSLLeftMargin}[1]{\parbox[t]{\csllabelwidth}{#1}}
\newcommand{\CSLRightInline}[1]{\parbox[t]{\linewidth - \csllabelwidth}{#1}}
\newcommand{\CSLIndent}[1]{\hspace{\cslhangindent}#1}

\begin{document}

% Print the header with above personal informations
% Give optional argument to change alignment(C: center, L: left, R: right)
\makecvheader

% Print the footer with 3 arguments(<left>, <center>, <right>)
% Leave any of these blank if they are not needed
% 2019-02-14 Chris Umphlett - add flexibility to the document name in footer, rather than have it be static Curriculum Vitae
\makecvfooter
  {julio 2022}
    {Alejandro Manuel Ferreiro~~~·~~~Curriculum Vitae}
  {\thepage~ of \pageref{LastPage}~}


%-------------------------------------------------------------------------------
%	CV/RESUME CONTENT
%	Each section is imported separately, open each file in turn to modify content
%------------------------------------------------------------------------------



\hypertarget{education}{%
\section{Education}\label{education}}

\begin{cventries}
    \cventry{National University of Córdoba}{PhD. Candidate, Biological Sciences}{Córdoba, Argentina}{2017-Present}{\begin{cvitems}
\item Title: Phylogeography and species distribucion models to define Conservation Units of the southern three-banded armadillo (Tolypeutes matacus) in Argentina.
\end{cvitems}}
    \cventry{National University of Córdoba}{Biologist}{Córdoba, Argentina}{2008-2014}{\begin{cvitems}
\item Title: Molecular phylogeny of the Euphractinae subfamily (Xenarthra). GPA 8.25/10
\end{cvitems}}
\end{cventries}

\hypertarget{journal-publications}{%
\section{Journal Publications}\label{journal-publications}}

\begin{cventries}
    \cventry{10.1111/jzs.12375}{Multiple refugia and glacial expansions in the Tucumane–Bolivian Yungas: The phylogeography and potential distribution modeling of Calomys fecundus (Thomas, 1926) (Rodentia: Cricetidae)}{2020}{Journal of Zoological Systematics and Evolutionary Research}{}\vspace{-4.0mm}
    \cventry{10.1371/journal.pone.0190944}{Phylogeography of screaming hairy armadillo Chaetophractus vellerosus: Successive disjunctions and extinctions due to cyclical climatic changes in southern South America}{2018}{PLoS ONE}{}\vspace{-4.0mm}
\end{cventries}

\hypertarget{other-publications}{%
\section{Other Publications}\label{other-publications}}

\hypertarget{bibliography}{}
\leavevmode\vadjust pre{\hypertarget{ref-Ferreiro2019}{}}%
\CSLLeftMargin{1. }
\CSLRightInline{Ferreiro, A. M., Abba, A. M., Camino, M., Tamburini, D.
M., Decarre, J., Soibelzon, E., Castro, L. B., Rogel, T. G., Agüero, A.
J., Albrecht, C. D., \& Superina, M. (2019). Tolypeutes matacus. In
\emph{Categorización 2019 de los mamíferos de argentina según su riesgo
de extinción. Lista roja de los mamíferos de argentina. Versión
digital}. SAyDS--SAREM (eds.). \url{http://cma.sarem.org.ar}}

\leavevmode\vadjust pre{\hypertarget{ref-Abba2019}{}}%
\CSLLeftMargin{2. }
\CSLRightInline{Abba, A. M., Camino, M., Torres, R. M., Ferreiro, A. M.,
Tamburini, D. M., Decarre, J., Soibelzon, E., Castro, L. B., Rogel, T.
G., Agüero, A. J., Albrecht, C. D., \& Superina, M. (2019).
Chaetophractus vellerosus. In \emph{Categorización 2019 de los mamíferos
de argentina según su riesgo de extinción. Lista roja de los mamíferos
de argentina. Versión digital}. SAyDS--SAREM (eds.).}

\hypertarget{manuscripts-submitted}{%
\section{Manuscripts submitted}\label{manuscripts-submitted}}

\begin{cventries}
    \cventry{Under second round of reviews}{Reconstructing the distribution of chacoan biota from current and past evidence: the case of the southern three-banded armadillo Tolypeutes matacus (Desmarest, 1804)}{}{Journal of Mammalian Evolution}{}\vspace{-4.0mm}
\end{cventries}

\hypertarget{awards-scholarships-fellowships-and-funding}{%
\section{Awards, Scholarships, Fellowships and
Funding}\label{awards-scholarships-fellowships-and-funding}}

\begin{cventries}
    \cventry{Fund for Scientific and Technological Research (FONCyT, Argentina).}{PICT project}{Research Fellow}{2022-2023}{\begin{cvitems}
\item Title: Biodiversity richness of non-flying mammals from central Argentina (La Pampa): changes in their distribution, dispersal pathways and extinctions during the Holocene. Project Leader: Esteban Soibelzon.
\end{cvitems}}
    \cventry{Fund for Scientific and Technological Research (FONCyT, Argentina)}{PICT project}{Research Fellow}{2021-2022}{\begin{cvitems}
\item Title: Evolutionary processes in mammals from the plains of Chaco, Espinal and the Pampean region: an approach from comparative phylogeography and ecological niche modeling. Project Leader: Raúl González Ittig.
\end{cvitems}}
    \cventry{IDEA WILD Foundation.}{IDEA WILD funding}{Project leader}{2019}{\begin{cvitems}
\item Title: Conservation of Southern three-banded Armadillo on its southernmost distribution range.
\end{cvitems}}
    \cventry{Argentinean Society for the study of Mammals (SAREM).}{SAREM Awards 2018}{Project leader}{2018}{\begin{cvitems}
\item Title: Phylogeography and species distribution modelling to define Conservation Units for the southern three-banded armadillo (Tolypeutes matacus).
\end{cvitems}}
    \cventry{Fund for Scientific and Technological Research (FONCyT, Argentina). Research Leader: Esteban Soibelzon.}{PICT project}{Research Fellow}{2017-2019}{\begin{cvitems}
\item Title: Distribution areas and dispersal routes of medium and small mammals during the Quaternary. The Arid South American Diagonal. Project leader: Esteban Soibelzon.
\end{cvitems}}
    \cventry{National University of Córdoba (Argentina)}{SECyT project}{Research Fellow}{2016-2017}{\begin{cvitems}
\item Title: Molecular systematics and phylogeography of mammals from southern South America: A study with epidemiological and conservation implications. Project leader: Marina Beatriz Chiappero.
\end{cvitems}}
    \cventry{CONICET (Argentina)}{PhD Fellowship}{Fellow}{2016-2022}{\begin{cvitems}
\item Title: Phylogeography and species distribution modelling to define Conservation Units for the southern three-banded armadillo (Tolypeutes matacus). Tutors: Marina Beatriz Chiappero and Esteban Soibelzon.
\end{cvitems}}
    \cventry{National Interuniversitary Council (CIN) (Argentina)}{Undergraduate scholarship (``Estimulo a las Vocaciones Cientificas'')}{Scholarship}{2013-2014}{\begin{cvitems}
\item Title: Molecular phylogeny of the Euphractinae subfamily (Xenarthra, Mammalia). Tutors: Sebastian Poljak and Raúl González Ittig.
\end{cvitems}}
\end{cventries}

\hypertarget{courses-taken}{%
\section{Courses taken}\label{courses-taken}}

\begin{cventries}
    \cventry{www.cognitiveclass.ai}{R101}{Online course.}{September 24th, 2021.}{\begin{cvitems}
\item Scoreless.
\end{cvitems}}
    \cventry{University of Kansas (KU)}{ENM 2020: Advanced course in Ecological Niche Modeling.}{Online course.}{From January to November, 2020.}{\begin{cvitems}
\item Scoreless.
\end{cvitems}}
    \cventry{Federal University of Paraná, Brazil.}{Cladistiscs, Phylosophy, Theory and Methods.}{Online course.}{July 6th-17th ,2020}{\begin{cvitems}
\item Lenght: 45hs. Score: B
\end{cvitems}}
    \cventry{National University of Córdoba, Argentina.}{Statistics}{Córdoba, Argentina}{October 29th to November 2nd, 2020}{\begin{cvitems}
\item Lenght: 40hs. Score: 9/10
\end{cvitems}}
    \cventry{National University of Tucumán, Argentina.}{Theoretical-practical advanced course in modeling of species niches, an approach to correlative and process-based models.}{Tucumán, Argentina}{October 21th-25th, 2019}{\begin{cvitems}
\item Lenght: 40hs. Score: 10/10
\end{cvitems}}
    \cventry{Argentinean Museum of Natural Sciences ¨Bernardino Rivadavia¨, Argentina.}{Extending and Enhancing DNA Barcoding Research in Argentina and Neighboring Countries: Tenth Leading Labs Training Workshop.}{Buenos Aires, Argentina}{September 17th-21th, 2018}{\begin{cvitems}
\item Lenght: 40hs. Scoreless.
\end{cvitems}}
    \cventry{www.conservationtraining.org}{IUCN Red List Assesment modules.}{Online course.}{Finished on March 15th, 2018.}{\begin{cvitems}
\item Scoreless.
\end{cvitems}}
    \cventry{National University of Tucumán, Argentina.}{Niche and distribution modeling, with an emphasis on the use of MAXENT}{Tucumán, Argentina}{July 10th-15th, 2017}{\begin{cvitems}
\item Lenght: 40hs. Score: 7/10
\end{cvitems}}
    \cventry{Red de Genética para la Conservación (ReGeneC) and the National University of Misiones, Argentina.}{XI Conservation Genetics Workshop}{Misiones, Argentina}{From January 29th to February 12th,  2017.}{\begin{cvitems}
\item Lenght: 100hs. Score: 83/100
\end{cvitems}}
    \cventry{University of Buenos Aires, Argentina.}{Genetic data analysis with R}{Buenos Aires, Argentina.}{From October 24th to November 2nd, 2016.}{\begin{cvitems}
\item Lenght: 60 hs. Score: 9/10.
\end{cvitems}}
    \cventry{National University of Cuyo, Argentina}{Conservation Biology}{Mendoza, Argentina}{September 9th-14th, 2016.}{\begin{cvitems}
\item Lenght: 45 hs. Score: 9/10.
\end{cvitems}}
    \cventry{National University of Córdoba, Argentina.}{Introduction to Phylogeography and Niche Modelling}{Córdoba, Argentina}{August 22th-26th, 2016.}{\begin{cvitems}
\item Lenght: 45 hs. Score: 10/10.
\end{cvitems}}
    \cventry{National University of Córdoba, Argentina.}{Analysis of the genetic structure of natural populations}{Córdoba, Argentina}{June 6th-10th, 2016.}{\begin{cvitems}
\item Lenght: 45 hs. Score: 10/10.
\end{cvitems}}
    \cventry{National University of Córdoba, Argentina.}{Introductory course to GIS and teledetection}{Córdoba, Argentina}{June 26th-27th, 2015.}{\begin{cvitems}
\item Lenght: 10 hs. Scoreless.
\end{cvitems}}
    \cventry{National University of Córdoba, Argentina.}{Biotechnology}{Córdoba, Argentina}{From March to June, 2013.}{\begin{cvitems}
\item Lenght: 40 hs. Score: 8/10.
\end{cvitems}}
\end{cventries}

\hypertarget{other-capacitations}{%
\section{Other capacitations}\label{other-capacitations}}

\begin{cventries}
    \cventry{Ministry of Science and Technology, Governement of the Province of Córdoba, Argentina}{i-Teams. Entrepreneurial training program and scientific-technological link}{Córdoba, Argentina}{From August to November, 2019}{}\vspace{-4.0mm}
\end{cventries}

\hypertarget{teaching-experience}{%
\section{Teaching experience}\label{teaching-experience}}

\begin{cventries}
    \cventry{National University of Córdoba}{Undergraduate assistant}{Córdoba, Argentina}{2011}{\begin{cvitems}
\item Course: Animal Diversity 1
\end{cvitems}}
    \cventry{National University of Córdoba}{Fellow colaboration}{Córdoba, Argentina}{2016-2018}{\begin{cvitems}
\item Course: Population Genetics and Evolution
\end{cvitems}}
\end{cventries}

\hypertarget{scientific-management-positions}{%
\section{Scientific management
positions}\label{scientific-management-positions}}

\begin{cventries}
    \cventry{Institute of Animal Diversity and Ecology (IDEA, CONICET-UNC)}{Representative of Fellows on the Board of Directors}{Córdoba, Argentina}{2019-2021}{}\vspace{-4.0mm}
\end{cventries}

\hypertarget{skills}{%
\section{Skills}\label{skills}}

\hypertarget{technical-skills}{%
\subsection{Technical skills}\label{technical-skills}}

\begin{cventries}
    \cventry{R, UNIX, JavaScript, HTML, CSS}{Coding languages}{}{}{}\vspace{-4.0mm}
    \cventry{QGIS, Mendeley, Zotero, Inkscape, PhotoShop, GitHub}{Softwares}{}{}{}\vspace{-4.0mm}
    \cventry{DNA extraction, PCR, Primer design, Electrophoresis}{Lab Skills}{}{}{}\vspace{-4.0mm}
\end{cventries}

\hypertarget{language-skills}{%
\subsection{Language skills}\label{language-skills}}

\begin{cventries}
    \cventry{Native language}{Spanish}{}{}{}\vspace{-4.0mm}
    \cventry{Advanced}{English}{}{}{}\vspace{-4.0mm}
\end{cventries}

\hypertarget{references}{%
\section{References}\label{references}}

\begin{cventries}
    \cventry{Institute of Animal Diversity and Ecology (IDEA; UNC, CONICET)}{Prof. Marina B. Chiappero}{}{}{\begin{cvitems}
\item Contact: marina.chiappero@gmail.com
\end{cvitems}}
    \cventry{Vertebrate Paleontology Division, La Plata Museum (UNLP, CONICET)}{Prof. Esteban Soibelzon}{}{}{\begin{cvitems}
\item Contact: esoibel@gmail.com
\end{cvitems}}
    \cventry{Southern Center for Scientific Research (CADIC; UNTDF, CONICET)}{Prof. Sebastian Poljak}{}{}{\begin{cvitems}
\item Contact: sebapoljak@hotmail.com
\end{cvitems}}
\end{cventries}


\label{LastPage}~
\end{document}
